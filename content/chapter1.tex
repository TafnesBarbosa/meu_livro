% \lettrine{}{} command typesets a drop cap in the document

% second brace set formats text placed inside as small caps
% (same thing as doing \lettrine{L}\textsc{orem ipsum ...}

\chapter{A floresta}
\lettrine{N}{o meio daquele planeta} havia um lugar muito conhecido, que por sua vez também era muito temido. Muitas histórias eram contadas e ouvidas a respeito deste lugar, histórias tenebrosas que faziam qualquer um tremer tanto as pernas que, hiperbolicamente, seria possível ouvir o som produzido por elas.

Um amontoado de árvores, pode-se dizer, para quem estava fora dele, mas uma infinidade delas para quem adentrava a floresta sombria. Esse era o nome pelo qual os habitantes de Gë a chamavam. Na verdade, era muito difícil encontrar alguém que soubesse a origem desse nome, mas não podia se negar a facilidade de se entender o porquê dele.

Podia-se encontrar todo e qualquer tipo de árvore dentro dela. Árvores grandes subindo até o ponto mais alto como também árvores troncudas e pequenas, que chegavam a ser engraçadas, mas que eram tão fortes como as outras. Além de folhas secas caídas das árvores, o chão era totalmente coberto por uma diversidade de vida. Em alguns lugares mais úmidos por um musgo misturado a lodo, em outros por uma grama que, inexplicavelmente, não parecia crescer o tanto que logicamente era esperado. Na verdade, havia uma explicação para isso: a floresta não era só composta de plantas, mas também de animais, tantos que não podiam ser contados, por mais que um leigo pudesse dizer o contrário levando em conta somente o que seus olhos vissem.

Semelhantemente às árvores, o ecossistema da floresta era diversificado, composto de animais de médio a pequeno porte e chegando até bichos invisíveis a olho nu, os mesmos que faziam qualquer contador gastar todos os seus dias e não terminar o seu intento. Isso que é naturalidade. 

Não se podia deixar enganar pela aparência dos animais, o mais inofensivo deles poderia ser o mais fatal e o mais amedrontador,... Bem, pelo rumo que essa história está tomando, dá pra perceber que eles seriam mais dóceis. Mas não, não era à toa que a floresta era temida. Havia algo ali que fazia com que tudo nela se tornasse mais selvagem e até mais maléfico.

E não eram só plantas e animais que viviam naquela floresta. Na verdade, nunca alguém houvera catalogado todos os seres vivos existentes ali. Mas, nos povoados ao redor dela, quando as pessoas se reuniam em torno de fogueiras, alguns típicos aventureiros costumavam falar que viram seres muito estranhos se movendo na borda da floresta, e isso quando eles não queriam chamar atenção para si mesmos falando que tinham presenciado um deles dentro dela. Mas quase nunca recebiam crédito, pois era conhecido que, uma vez lá dentro, raramente se era visto novamente.

Esse era um enigma de fato intrigante. Como uma floresta poderia comportar tantos imigrantes sendo que seu espaço físico não fazia jus à quantidade deles? Algo acontecia, com certeza. Alguns morriam e a própria terra se encarregava de abraçá-los, alguns poucos conseguiam sair, mas a outros algo místico deveria acontecer.